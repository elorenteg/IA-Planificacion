
\section{Implementación} \label{sec:implem}

En la \autoref{sec:modelo}, se ha descrito de forma detallada el modelo 
desarrollado para cada una de las variaciones del problema tratadas; estos 
modelos se encuentran implementados (expresados en el lenguaje \texttt{PDDL}) 
en el código adjunto a este documento. En esta sección, pues, se comenta el 
proceso seguido para su implementación. 

Para implementar el modelo final, se ha optado por un esquema de diseño 
incremental basado en prototipos, siguiendo el guión propuesto en el enunciado 
de la práctica. Es decir, se empezó con una solución para el problema básico 
propuesto, que sirvió esencialmente como toma de contacto con el lenguaje 
\texttt{PDDL} y la herramienta \texttt{Fast Forward}, y esta se fue ampliando 
poco a poco y de forma ordenada para tener en cuenta cada una de las 
extensiones trabajadas. Por lo tanto, a lo largo del desarrollo de la práctica 
se han ido construyendo prototipos, todos ellos perfectamente funcionales, 
cada vez teniendo en cuenta más elementos del problema hasta llegar a la 
última extensión. 

Sin embargo, la implementación del prototipo correspondiente a cada extensión 
partiendo de la anterior se ha llevado a cabo prácticamente en un solo paso 
(o sea, sin desarrollar diversos prototipos intermedios entre una extensión y 
la siguiente), puesto que la estructura en extensiones propuesta en el 
enunciado era suficientemente sencilla (es decir, los añadidos de una 
extensión respecto a la previa eran tan pocos que no justificaba una 
metodología más compleja).

Aparte de los modelos en \texttt{PDDL}, también se han desarrollado unos 
\textit{scripts} en \texttt{Python} (se encuentran también en el código adjunto 
a este documento): \texttt{generator.py} para la generación aleatoria de 
instancias de entrada de nuestro problema, \texttt{executor.py} para ejecutar 
un problema y mostrar toda la información extraída de la salida, 
\texttt{incrementator.py} para generar varios problemas con un cierto 
incremento en el número de tareas y número de programadores, y 
\texttt{genTablas.py}, que genera tablas y gráficos con los datos del programa
anterior. Todos estos \textit{scripts} se han utilizado para los experimentos 
descritos en la \autoref{sec:res-extra}. 


\clearpage

