
\subsection{Primera extensión} \label{sec:mod-ext1}

Esta primera extensión se construye a partir del problema básico añadiendo 
las tareas de revisión. Estas tareas de revisión, pues, provocan diversos 
cambios en el modelo que comentamos a continuación.

Para empezar, en esta versión no es suficiente saber que las tareas están 
asignadas a algún programador: hace falta, además, considerar si su revisión 
también se ha asignado. Además, la asignación de la tarea de revisión tiene 
algunas peculiaridades distintas (por ejemplo, hay que tener en cuenta que el 
programador que revisa una tarea no puede ser el mismo que la ha realizado).

En este modelo, no creamos un nuevo objeto de tipo \texttt{tarea} para 
representar la tarea de revisión. En vez de eso, consideramos un mayor número 
de estados en los que se puede encontrar una tarea con respecto al problema: 
una tarea puede estar no asignada para realizar, asignada para su realización 
pero no para su revisión o bien asignada tanto para su realización como para 
su revisión. Esto se modeliza añadiendo un nuevo predicado 
\texttt{(revisada ?t - tarea)} al modelo anterior. Se crea asimismo un 
predicado \texttt{(servida\_por ?t - tarea ?p - programador)} que indica el 
programador al cual se ha asignado la tarea (sin embargo, este nuevo predicado 
no sustituye a \texttt{(servida ?t - tarea)} por motivos de eficiencia, puesto 
que este último tiene un factor de ramificación menor); esto se utilizará para 
garantizar que la tarea de revisión la lleva a cabo un programador distinto.

De este modo, la acción \texttt{realiza} se modifica para incluir como efecto 
el uso de este nuevo predicado \texttt{servida\_por} para marcar qué 
programador realizará la tarea. Para esta extensión, además, es necesario 
crear una nueva acción para las tareas de revisión (puesto que ampliar la 
acción \texttt{realiza} para satisfacer también estas tareas de revisión la 
haría excesivamente compleja): se trata de la acción \texttt{revisa}. Esta 
admite como parámetros \texttt{(?t - tarea ?p - programador)} y se ejecuta 
cuando una tarea \texttt{?t} está servida pero no revisada y con un 
programador \texttt{?p} distinto del programador que realiza la tarea y que 
tiene suficiente habilidad para revisarla. En este caso, el efecto de la 
acción es marcar la tarea \texttt{?t} como revisada para indicar que la tarea 
de revisión asociada ya está asignada.



