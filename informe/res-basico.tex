
\subsection{Problema básico} \label{sec:res-basico}

En este problema solo se espera que se genere una asignación válida sin tener 
en cuenta el tiempo de ejecución de las tareas ni las tareas de revisión. Por 
lo tanto, las posibilidades a explorar en los juegos de prueba son muy 
limitadas. 

El primer juego de pruebas consiste en tres tareas, una de cada nivel de 
dificultad posible, y dos programadores con niveles de habilidad 1 y 2, 
respectivamente. En este caso, la única restricción es que la tarea de 
dificultad 3 tiene que asignarse al programador de habilidad 2. En este caso, 
el planificador encuentra una solución inmediatamente y sin dificultades.

En el segundo juego de pruebas, en cambio, se elimina el programador con 
nivel de habilidad 2, de manera que es imposible encontrar una asignación que 
satisfaga las restricciones del problema. El planificador es capaz de detectar 
esta situación e informa de la inexistencia de solución correctamente.





