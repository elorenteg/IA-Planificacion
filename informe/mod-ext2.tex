
\subsection{Segunda extensión} \label{sec:mod-ext2}

En esta extensión, se empieza a tener en cuenta el tiempo de realización de 
las tareas. Esto significa que hay que modificar las acciones adecuadamente 
para poder contabilizarlo y tener en cuenta las diferentes circunstancias que 
incrementan el tiempo de realización o de revisión (explicadas en la 
\autoref{sec:intro}).

Añadimos, pues, una nueva función llamada \texttt{(ttotal)} utilizada para 
almacenar el tiempo total de realización de tareas (incluyendo las de 
revisión) de la asignación parcial calculada hasta el momento. Esta función 
se utiliza como valor a minimizar en esta extensión. Además, su valor deberá 
ser actualizado con cada acción de asignación de tareas.

Por otro lado, en esta extensión no basta con saber que hay que realizar una 
tarea de revisión, sino que se necesita saber también cuánto tiempo requiere 
esta. El tiempo de revisión de una tarea depende de la calidad del programador 
que la haya desempeñado previamente. Por lo tanto, una posibilidad sería 
ampliar la acción \texttt{revisa} para que usase como parámetro el programador 
al cual se había asignado la tarea y calcular el tiempo de revisión según la 
calidad de este. Sin embargo, en este caso se ha optado por la introducción de 
dos nuevos predicados, \texttt{(por\_revisar\_1 ?t - tarea)} y 
\texttt{(por\_revisar\_2 ?t - tarea)} que indican que una tarea \texttt{?t} 
comporta una tarea de revisión de una hora o de dos, respectivamente. 

En particular, el efecto de la acción \texttt{realiza} debe adaptarse a estos 
cambios. Concretamente, cuando se ejecuta \texttt{realiza} con una tarea 
\texttt{?t} y un programador \texttt{?p}, se incrementa \texttt{ttotal} con 
el tiempo de realización de \texttt{?t} (incluyendo las dos horas adicionales 
por la habilidad de \texttt{?p} si cabe) y se activa uno de los dos predicados 
\texttt{por\_revisar\_1} o \texttt{por\_revisar\_2} según la calidad de 
\texttt{?p}.

La acción \texttt{revisa} también se adapta del mismo modo. Por un lado, 
cuando se unifica esta acción con una tarea \texttt{?t} y un programador 
\texttt{?p}, en la precondición se comprueba que uno de los dos predicados 
\texttt{(por\_revisar\_1 ?t)} o bien \texttt{(por\_revisar\_2 ?t)} se cumpla 
en vez de comprobar \texttt{(servida ?t)}; por otro lado, en el efecto se 
añade un incremento de \texttt{ttotal} en una o dos horas según el caso.




