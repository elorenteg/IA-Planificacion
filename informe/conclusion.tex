
\section{Conclusión} \label{sec:conclusion}

En este trabajo hemos resuelto un problema práctico modelizándolo como 
problema de planificación y utilizando un sistema de planificación automática
para su resolución. A pesar de tratarse de un problema relativamente simple, 
este problema nos ha permitido hacernos una idea del tipo de problemas a los 
que se pueden aplicar las técnicas de inteligencia artificial aprendidas y, 
en particular, de la capacidad expresiva de lenguajes formales como 
\texttt{PDDL} y la potencia de herramientas como \texttt{Fast Forward}.

En esta práctica, hemos analizado el problema de asignación de tareas de un 
proyecto a los programadores del equipo de desarrollo y hemos modelizado los 
elementos que intervienen en este problema mediante objetos, predicados y 
acciones de \texttt{PDDL}. Para ello, se ha seguido un desarrollo incremental 
guiado por las extensiones propuestas en el enunciado de la práctica. 

De este modo, hemos podido utilizar el planificador \texttt{Fast Forward} para 
solucionar instancias del problema debidamente seleccionadas y estudiar los 
resultados obtenidos. Así, pues, hemos podido comprobar como los sistemas de 
planificación automáticos diseñados de forma genérica y sin conocimiento del 
dominio son capaces de ofrecer buenas aproximaciones en tiempos de ejecución 
muy razonables, a pesar de su facilidad de uso y su potencia expresiva.


\clearpage

