
\subsection{Cuarta extensión} \label{sec:res-ext4}

%%% TODO

El primer juego de pruebas está compuesto por tres tareas y cinco 
programadores, con una selección de parámetros que da lugar a diferentes 
soluciones según se optimice sólo el tiempo total o sólo el número de 
programadores que han realizado/revisado alguna tarea. 

Por lo tanto, dado que esta extensión optimiza ambos criterios, podremos 
observar si logra obtener una solución ``intermedia'', que minimice tanto el 
tiempo total como el número de programadores.


%%% TODO: insertar salida?


Si optimizáramos sólo el tiempo, obtendríamos una planificación que tarda 21 
horas e involucra a 4 programadores; si, por el contrario, optimizáramos sólo 
el número de trabajadores, obtendríamos una planificación de 25 horas y 3 
programadores. Tal como se puede observar en la salida del programa, si 
minimizamos los dos criterios por igual (50\% y 50\%) conseguimos una solución 
con una duración de 21 horas (mejor tiempo) y 3 programadores involucrados 
(mejor número de trabajadores).



El segundo juego de pruebas pretende tratar el caso en que la ponderación de 
los dos criterios no sea la misma. Consta de cinco tareas y ocho 
programadores, de nuevo con parámetros seleccionados cuidadosamente para que se
obtengan diferentes soluciones cuando se optimizan los criterios por separado.

Optimizando los dos criterios equitativamente (50\% y 50\%) obtenemos una 
planificación con un tiempo total de 37 horas y 6 programadores, pero sabemos 
que esta solución no ha conseguido el número mínimo de programadores, que 
debería de ser 5 (uno para cada tarea y revisión, siempre que se cumplan los 
requisitos de habilidad -como es nuestro caso).

Hay que tener en cuenta que los dos criterios están expresados en unidades 
diferentes, pero como que no disponemos de la información para hacer una 
conversión de unidades ni el planificador permite optimizar expresiones 
lineares, aproximamos que el número de programadores debería de tener un peso 
del 80\%, y el tiempo total un 20\%. De esta manera, podemos obtener una 
planificación que tarda 37 horas y que sólo requiere a 5 programadores.
