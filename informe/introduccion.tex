
\section{Introducción y descripción del problema} \label{sec:intro}

Algunos problemas de optimización resultan computacionalmente intratables con 
los algoritmos exactos que se conocen a causa de su complejidad algorítmica, 
debida a la magnitud de la estructura combinatoria subyacente. Por ello, la 
inteligencia artificial utiliza técnicas que podan el espacio de búsqueda y 
lo inspeccionan guiadas por heurísticas para poder ofrecer soluciones 
aproximadas de forma sencilla. 

Una familia de problemas cuya resolución se ha simplificado mucho con 
herramientas genéricas que automatizan este proceso de búsqueda es la de los 
problemas de planificación. El objetivo de estos problemas es construir un 
plan o secuencia de acciones que permitan llegar a un cierto objetivo 
partiendo de un estado inicial. Se trata, pues, de problemas de síntesis, en 
los que hay que construir una solución basándose en el conocimiento del 
dominio (en este caso, en el estado del entorno, las acciones disponibles en 
cada momento y el objetivo al que se desea llegar).

La comunidad de inteligencia artificial ha desarrollado, a lo largo de los 
años, herramientas capaces de resolver problemas de planificación 
independientemente del dominio. Así, el usuario de estas herramientas solo 
tiene que modelizar su dominio y su problema mediante un lenguaje formal de 
representación y el sistema se encarga de la resolución automática del 
problema. 

En este trabajo, lo ejemplificamos mediante la resolución de un problema 
simple de asignación que se puede modelizar como un problema de planificación 
para guiar adecuadamente la construcción de la asignación. El problema 
consiste en la distribución de \(T\) tareas de programación entre \(P\) 
programadores. Cada tarea tiene asociada un nivel de dificultad (\(1\), \(2\) 
o \(3\), por orden creciente) y un tiempo estimado de desarrollo (un número 
de horas); por su lado, cada programador tiene asociada un nivel de habilidad 
(\(1\), \(2\) o \(3\), por orden creciente) y una categoría según su calidad 
(\(1\) o \(2\), por orden decreciente). Un programador puede resolver una 
tarea del mismo nivel de dificultad que su nivel de habilidad en el tiempo 
establecido o bien una tarea con un nivel de dificultad inmediatamente 
superior a su habilidad añadiendo dos horas adicionales. Tras la resolución 
de una tarea, se crea una tarea de revisión de una hora si el programador era 
de primera calidad o de dos horas en caso contrario, independientemente de si 
el programador que la lleva a cabo tiene una habilidad inferior; esta tarea de 
revisión tiene la misma dificultad que la inicial y tiene que ser llevada a 
cabo por otro programador distinto.

En esta práctica hemos considerado la versión básica del problema y las cuatro 
extensiones propuestas, de forma incremental. El problema básico consiste en 
la resolución directa del problema de asignación sin tener en cuenta las 
tareas de revisión. En la primera extensión, se añaden estas tareas de 
revisión al problema. La segunda extensión pretende, además, minimizar el 
tiempo total de resolución de las tareas. En la tercera extensión se limita 
el número de tareas por programador a \(2\) para facilitar el trabajo en 
paralelo. Finalmente, en la cuarta extensión, se pretende minimizar el 
producto entre el número de personas con tareas asignadas y el tiempo total de 
resolución.

Se desarrolla, pues, un modelo expresado en \texttt{PDDL} para cada uno de los 
problemas considerados, analizando los elementos necesarios y las diferencias 
entre ellos, y se resuelven utilizando la herramienta de planificación 
\texttt{Fast Forward}. El desarrollo de la práctica, así como la estructura 
de este informe, está influenciado por la metodología de la ingeniería del 
conocimiento vista en la práctica anterior, con un desarrollo incremental y 
basada en distintos prototipos (aunque en este caso, debido a la simplicidad 
del problema, las fases de identificación del problema, conceptualización y 
formalización prácticamente se han unido en una sola).

En el resto de este documento, explicamos los pasos del desarrollo de nuestra 
solución y mostramos y analizamos los resultados obtenidos.

\clearpage


