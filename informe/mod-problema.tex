
\subsection{Instancias del problema} \label{sec:mod-inst}

Hasta el momento se ha explicado como se modeliza el dominio de cada versión 
del problema. Del mismo modo, también hay que formalizar utilizando 
\texttt{PDDL} las entradas de los problemas a resolver.

Como se ha explicado, los únicos objetos del problema son las tareas y los 
programadores (cada uno con su propio tipo para reducir el número de posibles 
unificaciones). La entrada de un problema consiste, pues, en una enumeración 
de las tareas y de los programadores considerados seguida de una descripción 
de sus características. Es decir, se inicializan las funciones que 
corresponden a la dificultad y el tiempo necesario para cada tarea y a la 
habilidad y la calidad de cada programador. Esta parte de la descripción es 
común para todas las versiones del problema expuestas.

Además, algunas de las extensiones requieren de la inicialización de otras 
funciones que se utilizan para mantener guardados valores necesarios para 
construir la solución de forma adecuada. En particular, a partir de la segunda 
extensión hay que inicializar a cero el número de horas de realización de las 
tareas asignadas, a partir de la tercera extensión hay que inicializar el 
número de tareas asignadas a cada programador a cero y en la cuarta extensión 
hay que inicializar el número de programadores que trabajan en las tareas 
asignadas a cero, puesto que en la solución inicial no hay ninguna asignación. 

El estado inicial de la planificación, pues, es una asignación vacía. Esta 
asignación se va completando a medida que el planificador asigna tareas (tanto 
las tareas iniciales como las de revisión) a los programadores. Por lo tanto, 
una solución final se consigue cuando todas las tareas han sido asignadas a 
algún programador. Esto se modeliza en la versión básica del problema (en la 
cual no se consideran las tareas de revisión) pidiendo que todas las tareas 
estén servidas y en todas las extensiones posteriores pidiendo que todas las 
tareas estén revisadas (puesto que, por la definición de las acciones, una 
tarea solo puede estar revisada cuando también está servida). 

Finalmente, en algunas de las extensiones se pide que el planificador intente 
minimizar algunos criterios. Concretamente, en la segunda y en la tercera 
extensiones se intenta minimizar \texttt{(ttotal)} (que representa la suma de 
los tiempos de realización de todas las tareas, incluyendo las de revisión), 
mientras que en la cuarta extensión se intenta minimizar la suma de 
\texttt{(ttotal)} y \texttt{(ntrabajadores)} (que representa el número de 
programadores usados para realizar todas las tareas). 



