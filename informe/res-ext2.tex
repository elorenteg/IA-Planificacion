
\subsection{Segunda extensión} \label{sec:res-ext2}

En esta segunda extensión ya se impone un criterio de optimización. Esto 
permite explorar diversos casos construidos expresamente para estudiar si 
el planificador es capaz de encontrar el óptimo o, en caso contrario, cuán 
lejos se encuentra. 

Como no hay ninguna limitación en el número de tareas asignadas a cada 
programador, es fácil darse cuenta de que el óptimo de todos los problemas se 
obtiene cuando las tareas se asignan a los programadores con nivel de 
habilidad mayor o igual que las tareas, si los hay, y de entre estos a los 
programadores de primera calidad cuando sea posible (puesto que el incremento 
del tiempo debido a la habilidad del programador es de dos horas, mientras 
que el incremento del tiempo de revisión debido a la calidad del programador 
es solo de una hora). Esta observación nos permite calcular el óptimo 
``a mano'' en los casos de prueba propuestos y compararlo con la solución 
ofrecida por el planificador. De este modo, construimos dos juegos de 
pruebas para comprobar el efecto de la calidad y la habilidad de los 
programadores en la solución.

En el primero de ellos, se consideran tres tareas idénticas entre sí y dos 
programadores con la misma habilidad que pueden realizarlas, pero uno de 
ellos es de primera calidad y el otro no. En este caso, como era de esperar, 
la solución óptima consiste en asignar todas las tareas al programador de 
primera calidad y dejar las tareas de revisión para el otro programador. Esta 
es precisamente la solución encontrada por el planificador.

En el segundo juego de pruebas, en cambio, se utilizan dos programadores con 
la misma calidad pero distinta habilidad: uno de ellos requiere de dos horas 
adicionales para realizar las tareas, mientras que el otro puede realizarlas 
en el tiempo previsto. En este caso, pues, la solución óptima consiste en 
asignar todas las tareas al programador de mayor habilidad y dejar las 
revisiones para el otro programador. Otra vez, el planificador es capaz de 
encontrar la solución óptima.



