
\subsection{Problema básico} \label{sec:mod-basico}

Como se ha explicado en la \autoref{sec:intro}, los elementos principales del 
problema tratado son los programadores disponibles (con las características 
que definen sus habilidades) y las tareas que se deben asignar a estos 
(también con las propiedades que caracterizan su dificultad). Por este 
motivo, en nuestro modelo definimos dos nuevos tipos: \texttt{tarea} y 
\texttt{programador} (ambos subtipos de \texttt{object}). Esto permite al 
sistema planificador ser más eficiente, pues se restringen las posibilidades 
de unificación y se reduce así el factor de ramificación.

Las propiedades que caracterizan las tareas y los programadores en este 
problema son números enteros con los que es necesario realizar operaciones 
aritméticas o comparaciones (por ejemplo, para determinar si un programador 
dispone de la habilidad suficiente para realizar una tarea). Se ha decidido, 
por lo tanto, crear funciones para definirlas. En particular, se utilizan las 
funciones \texttt{(dtarea ?t - tarea)} y \texttt{(ttarea ?t - tarea)} para 
obtener el nivel de dificultad y el tiempo mínimo necesario para la 
realización, respectivamente, de una tarea \texttt{?t} y las funciones 
\texttt{(hprog ?p - programador)} y \texttt{(cprog ?p - programador)} para 
obtener el nivel de habilidad y la calidad, respectivamente, de un programador 
\texttt{?p}. 

Para esta versión básica, no se consideran las tareas de revisión inducidas 
por las tareas iniciales del problema. Así, pues, las tareas se encuentran en 
exactamente uno de dos estados posibles en el problema: no asignadas o 
asignadas. Para modelizar estas posibilidades, se utiliza un predicado 
\texttt{(servida ?t - tarea)} que deberá cumplirse cuando la tarea \texttt{?t} 
esté asignada. Este predicado es necesario para saber cuándo se ha llegado a 
una solución (es decir, todas las tareas están asignadas) y cuándo se puede 
asignar una tarea a algún programador para acercarse a una solución.

Finalmente, en esta versión básica la única acción necesaria para llevar a 
cabo una asignación completa de las tareas a los programadores es asignar una 
tarea \texttt{?t} no asignada previamente a un programador \texttt{?p} capaz 
de ejecutarla. A tal efecto, definimos una acción \texttt{realiza} que tiene 
como parámetros \texttt{(?t - tarea ?p - programador)} y, si se cumple que la 
tarea \texttt{?t} no se ha asignado todavía y que el programador \texttt{?p} 
tiene la habilidad suficiente para realizarla, la tarea se le asigna y se 
marca como servida.

Observamos con esta explicación que algunos de los atributos que definen las 
tareas y los programadores (funciones en este modelo) no se utilizan. Aún así, 
estos son necesarios para las siguientes versiones del problema y, 
manteniéndolos en este modelo, permitimos que la entrada del problema se pueda 
definir prácticamente del mismo modo (es decir, una entrada para una versión 
del problema podría servir para otra versión con modificaciones menores). 
Esto nos es especialmente útil para las instancias del problema generadas 
automáticamente.



