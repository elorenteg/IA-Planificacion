
\subsection{Cuarta extensión} \label{sec:mod-ext4}

En esta última extensión se intenta minimizar la suma ponderada del número 
total de programadores que trabajarán más el tiempo total invertido en la 
realización del proyecto. Con ello, se pretende minimizar ambos factores a la 
vez.

Para implementar esta versión, pues, hay que controlar qué programadores 
trabajan y cuántos son. Se añaden así un predicado 
\texttt{(trabaja ?p - programador)} para marcar cuando un cierto programador 
\texttt{?p} tiene asignada al menos una tarea y una función 
\texttt{(ntrabajadores)} para mantener el número de programadores empleados 
para la asignación de tareas hasta el momento. 

Las acciones \texttt{realiza} y \texttt{revisa} actualizan sus efectos 
acordemente y, por lo tanto, cuando se aplican con un programador \texttt{?p} 
que hasta el momento no trabajaba, se marca que este trabaja y se incrementa 
\texttt{ntrabajadores} en una unidad.




